\documentclass[12pt,a4paper]{scrartcl}
\usepackage[utf8]{inputenc}
\usepackage[english,russian]{babel}
\usepackage{indentfirst}
\usepackage{misccorr}
\usepackage{graphicx}
\usepackage{amsmath}
\begin{document}
\begin{titlepage}
\begin{center}
\large

Дискретное преобразование Фурье.

\end{center}
\vfill
\begin{flushright}
Лукьянчиков Иван

Группа 424
\end{flushright}
\end{titlepage}
\textbf{1) Постановка задачи.}
Имеем разностное уравнение
$$\dfrac{y_{k+1}-2y_{k}+y_{k-1}}{h^2}=-\lambda y_k, \quad h=1/N, \quad 1 \leqslant k \leqslant N-1$$
$$y_0=y_N=0$$
Требуется анлитически найти $\lambda^{(m)}$ и $y_k^{(m)}$, затем численно посчитать невязку, проверить ортогональность системы $\{y_k^{(m)}\}$ и разложить функцию $f$ удовлетворяющую начальным условиям через линейную комбинацию $\{y_k^{(m)}\}$.

\textbf{2) Решение задачи.}
Характеристическое уравнение задачи имеет вид
$$\mu^2 - p\mu + 1 = 0, \quad \text{где} \quad  p=2 - h^2 \lambda$$
$$\mu_{1,2} = \dfrac{p \pm \sqrt{p^2 - 4}}{2}$$

Если корни характеристического уравнения различны (равные корни рассматириваются аналогично) и вещественны, то общее решение имеет вид $$y_k = C_1 \mu_1^k + C_2 \mu_2^k$$ откуда из краевых условий получаем $$C_1 + C_2 = 0, \   C_1 \mu_1^N + C_2 \mu_2^N = 0 $$ откуда следует, что $y_k = C_1 \mu_1^N - C_1 \mu_1^N = 0$. Так как $\mu_1 \neq \mu_2$, то $C_1 = C_2 = 0$.

Поэтому следует рассматривать случай комплексно-сопряженных корней $\mu_{1,2} = cos\phi + isin\phi$. Тогда общее решение задачи имеет вид

$$y_k = C_1 cos(k \phi) + C_2 sin(k \phi)$$

Из краевых условий получаем $C_1 = 0$ и $sinN\phi = 0$, отсюда $\phi = \pi m / N, m=0,\pm1,\pm2,...$ подставляя $k = 0$ и $k = -N$ получим, что $\phi = 2 \pi m / N$ Так как $\mu_1 + \mu_2 = 2 - h^2 \lambda$, то $cos \phi = 1 - \dfrac{h^2 \lambda}{2}$, следовательно

$$\lambda^{(m)} = \dfrac{4}{h^2} \; sin^2 \dfrac{\pi m}{N}$$

и общее решение имеет вид

$$y_k^{(m)} = C_2 sin \dfrac{\pi m k}{N} $$

Рассмотрим вектора $y^i = (y_0^{i}, \dots y_N^{i})$, где $i = 1, \dots, N-1$

Рассмотрим произвольную функцию, удовлетворяющую граничным условиям. Подставляя в функцию значения $hk$,$k=0,1, \dots ,N$ получаем вектор f и находим его разложение в базисе ${y^i}$. В пространстве со скалярным произведением $\sum_{k=1}^{N-1} v_i u_i h + 0.5h[v_0 u_0+v_N u_N]$, $y^i$ образуют ортонормированный базис. Имеем $f = \sum_{i=1}^{N-1} d_i y^i $, где $d_i = (f,y^i)$.

Пусть теперь теперь количество точек увелеченно, например $M = 2N-1$. Получем вектор $f$ для $M$ точек. Это значение может быть получено двумя способами:

1. при помощи суммы базисных векторов с найденными ранее коэффициентами;

2. явно подставив точки в выражение, для ранее заданной произвольной функции, удовлетворяющей граничным условиям.

Для проверки корректности приближения(разложение является приближением функции по N точкам) находим норму разности векторов, полученных двумя способами.

\textbf{3) Реализация задачи.}
Задача была реализована на языке програмированния С++. Мною был разработан class myvector и в рамках этого класса были реазлизованны следующие функции: сложение векторов, умножение вектора на скаляр, и скалярное произведение векторов. И все посчитанно.

\textbf{4) Тестирование.}
Было численно проверена ортогональность системы $y^i$ для $N=101,1001$, через delta обозначено максимальное отклонение от 0, i и j вектора, на которых достигается маскимально отклонение.

N=101:

delta=6.25311e-15 i=97 j=98

N=1001:

delta=1.66993e-13 i=999 j=1000

Разложение функции по базисным векторам было протестированно на следующих данных. На отрезке $[0;1]$ была взята равномерная сетка из 2N-1 узлов, в качестве приближаемой функции $y=cos(2\pi x) - 1$, которая приближалась по системе из N функций вида $C_2 sin ((\pi m k)/(2N-1))$. Для оценки погрешности были посчитанны различные нормы от вектора разности.

N=101:

diff=2.06715e-10 diff=4.13429e-08 diff=4.30862e-05

N=1001:

diff=2.06709e-15 diff=4.13417e-12 diff=4.30922e-07

\end{document}