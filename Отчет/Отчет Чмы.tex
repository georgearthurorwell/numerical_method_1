\documentclass[12pt,a4paper]{scrartcl}
\usepackage[utf8]{inputenc}
\usepackage[english,russian]{babel}
\usepackage{indentfirst}
\usepackage{misccorr}
\usepackage{graphicx}
\usepackage{amsmath}
\begin{document}
\begin{titlepage}
\begin{center}
\large

Дискретное преобразование Фурье.

\end{center}
\vfill
\begin{flushright}
Лукьянчиков Иван

Группа 424
\end{flushright}
\end{titlepage}
\textbf{1) Постановка задачи.}
Имеем разностное уравнение
$$\dfrac{y_{k+1}-2y_{k}+y_{k-1}}{h^2}=-\lambda y_k, \quad h=1/N, \quad 1 \leqslant k \leqslant N-1$$
$$тут граничные услоивия$$
Требуется анлитически найти $\lambda^{(m)}$ и $y_k^{(m)}$, затем численно посчитать невязку, проверить ортогональность системы $\{y_k^{(m)}\}$ и разложить функцию $f$ удовлетворяющую начальным условиям через линейную комбинацию $\{y_k^{(m)}\}$.

\textbf{2) Решение задачи.}
Характеристическое уравнение задачи имеет вид
$$\mu^2 - p\mu + 1 = 0, \quad \text{где} \quad  p=2 - h^2 \lambda$$
$$\mu_{1,2} = \dfrac{p \pm \sqrt{p^2 - 4}}{2}$$

\textbf{3) Реализация задачи.}
Задача была реализована на языке програмированния С++. Мною был разработан class myvector и в рамках этого класса были реазлизованны следующие функции: сложение векторов, умножение вектора на скаляр, и скалярное произведение векторов. И все посчитанно.

\textbf{4) Тестирование.}
Реализованная мной программа была протестированна на следующих данных. На отрезке $[-1;1]$ была взята равномерная сетка из 101 узла, в качестве приближаемой функции $y=\dfrac{1}{25x^2+1}$, которая приближалась по системе из 11 функций вида $cos(\pi*n*x)$, где $n=0,1,\dots,10$. Результаты будут приведены ниже на графике.
\end{document}